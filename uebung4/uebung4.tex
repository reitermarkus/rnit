% !TEX TS-program = xelatex
%
\documentclass[ngerman]{scrartcl}

\usepackage[ngerman]{babel}
\usepackage{amsmath}
\usepackage{enumitem}

\usepackage{siunitx}
\sisetup{per-mode = symbol, binary-units = true}
\DeclareSIUnit\mbyte{Byte}

\usepackage{stackengine}
\usepackage[usenames, dvipsnames]{xcolor}

\begin{document}

\section*{1}
	\begin{enumerate}[label=\alph*)]
		\item
		T: \ \ \ \ Twisted Pair \\
		10: \ \ \ Übertragungsrate von \SI{10}{\mega\bit\per\s} \\
		Base: Basisband \\
		5: \ \ \ \ \ Maximale Leitungslänge von 500 Meter \\
		\newline
		Medienzugriffskontrolle: CSMA/CD \\
		Kodierung: Manchester-Code \\
		Minimale Rahmengröße: 64 Byte \\
		Maximale Rahmengröße: 1518 Byte \\

		\item
		Signalgeschwindigkeit = Lichtgeschwindigkeit $\cdot$ Ausbreitungsfaktor (0.77) \\
		\newline
		theoretische minimale Rahmengröße:
		\begin{itemize}
		  \item doppelte Übertragungsrate aus Toleranzgründen (Verzögerungen usw.)
		  \item doppelte Länge für Hin und Rückweg.
		\end{itemize}
		\begin{align*}
		  2 \cdot \text{Übertragungsrate} \cdot \dfrac{2 \cdot \text{Länge}}{\text{Signalgeschwindigkeit}}
		\end{align*}
		\begin{align*}
		  \SI{20000000}{\bit\per\s} \cdot \dfrac{\SI{5000}{\m}}{\SI{231000000}{\ms}}
		\end{align*}
		\begin{align*}
		  \SI{20000000}{\bit\per\s} \cdot 0,00002164 = \SI{432}{\bit} = \SI{54}{\mbyte}
		\end{align*}
		\item
		10G: \  \ Übertragungsrate von \SI{10}{\giga\bit\per\s} \\
		Base: \ Basisband \\
		T: \ \ \ \ \ Twisted Pair \\
		\newline
		Im Vergleich zu T10Base5:
		\begin{itemize}
			\item kürzere Reichweite (\SI{100}{\m})
			\item höhere Übertragungsreichweite (\SI{10}{\giga\bit\per\s})
			\item andere Kodierung: 128-DSQ, LDPC
		\end{itemize}
		\item
		Nach fünf 1en wird immer eine 0 eingefügt. \\
		01111110 markiert Anfang und Ende. \\
		\newline
		01111\underline{01111110}0000011111{\textcolor{ForestGreen}{\textbf{0}}}01011111{\textcolor{ForestGreen}{\textbf{0}}}1 \\
	\end{enumerate}

\section*{2}

	\begin{enumerate}[label=\alph*)]
	\item
	\begin{tabular}{cc}
		\begin{minipage}{.5\linewidth}
			\begin{enumerate}[label=\roman*)]
				\item
				\begin{tabular} {c c c c c c }
					1 & 1 & 0 & 1 & \  & 0                             \\
					1 & 0 & 1 & 1 & \  & 1                             \\
					1 & 1 & 0 & 1 & \  & 1                             \\
					1 & 0 & 1 & 0 & \  & 0                             \\
					  &   &   &   &    &                               \\
					1 & 0 & 0 & 1 &    &                               \\
				\end{tabular}
				\newline
				\newline
				\item
				\begin{tabular} {c c c c c c }
					0 & 0 & 1 & 1 & \  & 0                             \\
					0 & 1 & 0 & 1 & \  & {\textcolor{red}{\textbf{1}}} \\
					1 & 0 & 1 & 0 & \  & {\textcolor{red}{\textbf{1}}} \\
					1 & 0 & 1 & 1 & \  & 1                             \\
				 	  &   &   &   &    &                               \\
					0 & 0 & 1 & 0 &    &                               \\
				\end{tabular}
				\newline
				\newline
				\item
				\begin{tabular} {c c c c c c }
					1 & 0 & 1 & 1 & \  & 1                             \\
					0 & 1 & 1 & 1 & \  & {\textcolor{red}{\textbf{1}}} \\
					1 & 0 & 1 & 0 & \  & 0                             \\
					1 & 1 & 0 & 0 & \  & 0                             \\
					  &   &   &   &    &                               \\
					1 & {\textcolor{red}{\textbf{1}}} & 1 & 0 &   &    \\
				\end{tabular}
			\end{enumerate}
		\end{minipage} &

		\begin{minipage}{.5\linewidth}
			\begin{enumerate}[label=\roman*)]
				\item
				\begin{tabular} {c c c c c c }
					1 & 1 & 0 & 1 & \  & 0                             \\
					1 & 0 & 1 & 1 & \  & 1                             \\
					1 & 1 & 0 & 1 & \  & 1                             \\
					1 & 0 & 1 & 0 & \  & 0                             \\
					  &   &   &   &    &                               \\
					1 & 0 & 0 & 1 &    &                               \\
				\end{tabular}
				\newline
				\newline
				\item
				\begin{tabular} {c c c c c c }
					0 & 0 & 1 & 1 & \  & 0                             \\
					0 & 1 & 0 & {\textcolor{ForestGreen}{\textbf{0}}} & \ & 1\\
					1 & 0 & 1 & {\textcolor{ForestGreen}{\textbf{1}}} & \ & 1\\
					1 & 0 & 1 & 1 & \  & 1                             \\
					  &   &   &   &    &                               \\
					0 & 0 & 1 & 0 &    &                               \\
				\end{tabular}
				\newline
				\newline
				\item
				\begin{tabular} {c c c c c c }
					1 & 0 & 1 & 1 & \  & 1                              \\
					0 & 1 & 1 & 1 & \  & {\textcolor{ForestGreen}{\textbf{1}}}\\
					1 & 0 & 1 & 0 & \  & 0                              \\
					1 & 1 & 0 & 0 & \  & 0                              \\
					  &   &   &   &    &                                \\
					1 & {\textcolor{ForestGreen}{\textbf{1}}} & 1 & 0 & &     \\
				\end{tabular}
			\end{enumerate}
		\end{minipage}
	\end{tabular}
  \item
  Vorgehensweise

  \begin{enumerate}[label=\arabic*)]
    \item Von links nach rechts, mit der ersten stelle beginnend, werden die Stellen sich wiederholend mit 7, 3, 1 gewichtet.
    \item Die jeweiligen Produkte aus beiden Zahlen werden errechnet.
    \item Die Endziffern der Produkte werden summiert.
    \item Die Prüfsumme ist die letzte Ziffer der Summe.
  \end{enumerate}
  \pagebreak
  Prüfsumme für Seriennummer \\

  \begin{tabular}{|*{4}{c |}}
    \hline
    NR & Gewichtung & Produkt & Endziffern     \\
    \hline
    4  & 7          & 28      & 8              \\
    2  & 3          & 6       & 6              \\
    1  & 1          & 1       & 1              \\
    0  & 7          & 0       & 0              \\
    1  & 3          & 3       & 3              \\
    0  & 1          & 0       & 0              \\
    1  & 7          & 7       & 7              \\
    0  & 3          & 0       & 0              \\
    \hline
       &            &         & 2\underline{5} \\
    \hline
  \end{tabular} \\

  Prüfsumme für Geburtsdatum \\

  \begin{tabular}{|*{4}{c |}}
    \hline
    NR & Gewichtung & Produkt & Endziffern     \\
    \hline
    8  & 7          & 56      & 6              \\
    7  & 3          & 21      & 1              \\
    0  & 1          & 0       & 0              \\
    7  & 7          & 49      & 9              \\
    2  & 3          & 6       & 6              \\
    7  & 1          & 7       & 7              \\
    \hline
       &            &         & 2\underline{9} \\
    \hline
  \end{tabular} \\

  Prüfsumme für Ablaufdatum

  Heiner hat die Gesamtprüfziffer 4 vorgegeben, deshalb muss die Prüfsumme beim
  Ablaufdatum 0 sein. $5 + 9 = 1\underline{4}$. Der Ausweis wurde im Jahr 2006
  ausgestellt und ist 10 Jahre lang valide. Für die längst mögliche Gültigkeit
  wurde das Monat Dezember gewählt. Der Tag wird im folgendem noch errechnet. \\

  \begin{tabular}{|*{4}{c |}}
    \hline
    NR & Gewichtung & Produkt & Endziffern                 \\
    \hline
    1  & 7          & 7       & 7              \\
    6  & 3          & 18      & 8              \\
    1  & 1          & 1       & 1              \\
    2  & 7          & 14      & 4              \\
    3  & 3          & 9       & 9              \\
    1  & 1          & 1       & 1              \\
    \hline
       &            &         & 3\underline{0} \\
    \hline
  \end{tabular} \\

  Ablaufdatum 31.12.2016. Die Gesamtprüfziffer ist immer noch 4. ($5 + 9 + 0 = 1\underline{4}$)

  \item
  \begin{itemize}
    \item Ausweis ist nur bis 2016 gültig, daher veraltet.
    \item unrealistisches Foto.
    \item eventuell entspricht das Geburtsdatum nicht seinem Erscheinungsbild.
  \end{itemize}
\end{enumerate}


\section*{3}

\begin{enumerate}[label=\alph*)]
  \item
  \(
    x^6 + x^4 + x^3 + 1 : x^3 + 1 = x^6 + x^4 + x + 1
    \\
    x + 1\ \text{Rest}
    \\
    \\
    \stackMath\def\stackalignment{l}
    \stackunder{%
      1011001000 : 1001 = 1010011
    }{%
      \Shortstack[l]{
        {     \underline{1001}}
        {\textcolor{white}{1}           0100}
        {\textcolor{white}{1}\underline{0000}}
        {\textcolor{white}{12}           1000}
        {\textcolor{white}{12}\underline{1001}}
        {\textcolor{white}{123}           0011}
        {\textcolor{white}{123}\underline{0000}}
        {\textcolor{white}{1234}           0110}
        {\textcolor{white}{1234}\underline{0000}}
        {\textcolor{white}{12345}           1100}
        {\textcolor{white}{12345}\underline{1001}}
        {\textcolor{white}{123456}           1010}
        {\textcolor{white}{123456}\underline{1001}}
        {\textcolor{white}{1234567}           011\ \text{Rest}}
      }
    }
  \)

  \addtocounter{enumi}{3}
  \item
  $(x + 1)\ |\ g(x) \Leftrightarrow 11\ \text{über}\ GF(2)$

  Bei Division durch 11 wird $00$ an den Dividenden angehängt. Der Rest von $0$ entsteht
  durch Polynomdivision über $GF(2)$ bei ungerader Bitanzahl. Wenn jedoch der CRC-Wert
  nicht 00 ist kann der Rest auch nicht 0 sein und deshalb ist nur ein gültiger CRC-Wert
  möglich.
\end{enumerate}


\section*{4}

\begin{enumerate}[label=\alph*)]
  \item

  Die Übertragunszeit für \SI{1}{\kilo\bit} beträgt \SI{0,001}{\second} +
  \SI{0,25}{\second} Latenz.

  ACK benötigt ca. \SI{0,25}{\second} (\SI{3}{\bit} + Latenz).

  Stop-and-Wait
  \begin{itemize}
    \item Sender sendet Rahmen.
    \item Empfänger sendet ACK.
    \item Sender empfängt ACK.
  \end{itemize}

  Die Übertragungkapazität beträgt ca. \SI{2}{Rahmen\per\second} $\approx$
  \SI{2}{\kilo\bit\per\second}
  \pagebreak
  \item
  Pipelining
  \begin{itemize}
    \item Sender sendet Rahmen sequentiell.
    \item Empfänger sendet ACKs parallel zurück.
    \item Fehlerhafte Rahmen werden neu gesendet.
  \end{itemize}

  Sendezeit = \SI{0,5}{\second}

  \begin{itemize}
    \item \SI{0,25}{\second} Latenz vor Empfang des ersten Rahmens
    \item \SI{0,25}{\second} Latenz vor Empfang des ersten ACK
  \end{itemize}

  Ein Rahmen benötige \SI{0,001}{\second} und ein ACK \SI{0,000003}{\second}.

  Annahme: 1000 Rahmen + 1000 ACK = \SI{1,003}{\second}

  Die Übertragungkapazität beträgt ca. \SI{500}{Rahmen\per\second} $\approx$
  \SI{500}{\kilo\bit\per\second}
\end{enumerate}

\end{document}
