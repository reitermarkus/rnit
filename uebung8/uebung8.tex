% !TEX TS-program = xelatex
%
\documentclass[ngerman]{scrartcl}

\usepackage[ngerman]{babel}
\usepackage{amsmath}
\usepackage{enumitem}

\usepackage{siunitx}
\usepackage{eurosym}
\sisetup{per-mode = symbol, binary-units = true}
\DeclareSIUnit\mbyte{Byte}

\usepackage{setspace}% http://ctan.org/pkg/setspace
\usepackage{lipsum}% http://ctan.org/pkg/lipsum

\begin{document}

\section*{1}
  \begin{enumerate}[label=\alph*)]
    \item
    \begin{enumerate}[label=\Roman*)]
      \item
      \begin{tabular}{|*{3}{c|}}
        \hline
        $v_i$ & $B_i$ & $O_i$ \\
        \hline
        A & \{B, D\} & \{B\}    \\
        \hline
        B & \{A, D\} & \{A, C\} \\
        \hline
        C & \{B\}    & \{D, E\} \\
        \hline
        D & \{C, E\} & \{A, B\} \\
        \hline
        E & \{C\}    & \{D\}    \\
        \hline
      \end{tabular}
      \begin{doublespacing}
        \begin{tabular}{|l|*{5}{c|}}
          \hline
          Iteration & A & B & C & D & E \\
          \hline
          1 & $\frac{1}{5}$ & $\frac{1}{5}$ & $\frac{1}{5}$ & $\frac{1}{5}$ & $\frac{1}{5}$ \\
          \hline
          2 & $\frac{1}{5}$ & $\frac{3}{10}$ & $\frac{1}{10}$ & $\frac{3}{10}$ & $\frac{1}{10}$ \\
          \hline
          3 & $\frac{3}{10}$ & $\frac{7}{20}$ & $\frac{3}{20}$ & $\frac{3}{20}$ & $\frac{1}{20}$ \\
          \hline
          4 & $\frac{1}{4}$ & $\frac{3}{8}$ & $\frac{7}{40}$ & $\frac{1}{8}$ & $\frac{3}{40}$ \\
          \hline
          5 & $\frac{1}{4}$ & $\frac{5}{16}$ & $\frac{3}{16}$ & $\frac{13}{80}$ & $\frac{7}{80}$ \\
          \hline
          6 & $\frac{19}{80}$ & $\frac{53}{160}$ & $\frac{5}{32}$ & $\frac{29}{160}$ & $\frac{3}{32}$ \\
          \hline
          7 & $\frac{41}{160}$ & $\frac{21}{64}$ & $\frac{53}{320}$ & $\frac{11}{64}$ & $\frac{5}{64}$ \\
          \hline
          8 & $\frac{1}{4}$ & $\frac{219}{640}$ & $\frac{21}{128}$ & $\frac{103}{640}$ & $\frac{53}{640}$ \\
          \hline
        \end{tabular}
      \end{doublespacing}

      \item
      \begin{tabular}{|*{3}{c|}}
        \hline
        $v_i$ & $B_i$ & $O_i$    \\
        \hline
        A & \{E\}     & \{B, C\} \\
        \hline
        B & \{A\}     & \{C\}    \\
        \hline
        C & \{A, B\}  & \{D\}    \\
        \hline
        D & \{C\}     & \{E\}    \\
        \hline
        E & \{D\}     & \{A\}    \\
        \hline
      \end{tabular}
      \begin{doublespacing}
        \begin{tabular}{|l|*{5}{c|}}
          \hline
          Iteration & A & B & C & D & E \\
          \hline
          1 & $\frac{1}{5}$ & $\frac{1}{5}$ & $\frac{1}{5}$ & $\frac{1}{5}$ & $\frac{1}{5}$ \\
          \hline
          2 & $\frac{1}{5}$ & $\frac{1}{10}$ & $\frac{3}{10}$ & $\frac{1}{5}$ & $\frac{1}{5}$ \\
          \hline
          3 & $\frac{1}{5}$ & $\frac{1}{10}$ & $\frac{1}{5}$ & $\frac{3}{10}$ & $\frac{1}{5}$ \\
          \hline
          4 & $\frac{1}{5}$ & $\frac{1}{10}$ & $\frac{1}{5}$ & $\frac{1}{5}$ & $\frac{3}{10}$ \\
          \hline
          5 & $\frac{3}{10}$ & $\frac{1}{10}$ & $\frac{1}{5}$ & $\frac{1}{5}$ & $\frac{1}{5}$ \\
          \hline
          6 & $\frac{1}{5}$ & $\frac{3}{20}$ & $\frac{1}{4}$ & $\frac{1}{5}$ & $\frac{1}{5}$ \\
          \hline
          7 & $\frac{1}{5}$ & $\frac{1}{10}$ & $\frac{1}{4}$ & $\frac{1}{4}$ & $\frac{1}{5}$ \\
          \hline
          8 & $\frac{1}{5}$ & $\frac{1}{10}$ & $\frac{1}{5}$ & $\frac{1}{4}$ & $\frac{1}{4}$ \\
          \hline
        \end{tabular}
      \end{doublespacing}
    \end{enumerate}
  \end{enumerate}

\section*{2}
	Festplatte Seagate SkyHawk mit 10tb Speicherkapazität kostet 365€ und wiegt 649g (Daten Amazon)
	\begin{enumerate}[label=\alph*)]
		\item 
		\begin{align*}
			\SI{5,76}{\peta\byte} = \SI{5760}{\tera\byte}
		\end{align*}
		\begin{align*}
			\frac{5760}{10} = 576 \text{ Festplatten}
		\end{align*}
		\begin{align*}
			576 \cdot \SI{649}{\gram} = \SI{373824}{\gram} = \SI{373,824}{\kilogram}
		\end{align*}
		Alle Festplatte passen also in einen LKW.
		
		\item 
		\begin{align*}
			365 \euro \cdot 576 = 210240\euro
		\end{align*}
				
		\item 
		Entfernung von Innsbruck nach Trient beträgt \SI{177}{\kilo\metre}. Durchschnittsgeschwindigkeit von \SI{60}{\kilo\metre\per\hour} würde die Fahrt ca \SI{3}{\hour} dauern. 
		\begin{align*}
			\SI{3}{\hour} = \SI{10800}{\second}			
		\end{align*}
		\begin{align*}
			\frac{\SI{5,76}{\peta\byte}}{\SI{10800}{\second}} = \SI{0,5333}{\tera\byte\per\second}		
		\end{align*}
	\end{enumerate}
\end{document}
