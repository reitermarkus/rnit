% !TEX TS-program = xelatex
%
\documentclass[ngerman]{scrartcl}

\usepackage[ngerman]{babel}
\usepackage{amsmath}
\usepackage{enumitem}

\usepackage{siunitx}
\sisetup{per-mode = symbol, binary-units = true}


\begin{document}

\section*{1}

\begin{enumerate}[label=\alph*)]
  \item
  \begin{align*}
    \text{normalisierter Durchsatz S} &= \frac{\SI{2000}{\bit}}{\SI{50}{\s}} = \SI{40}{\bit\per\s}\\
  \end{align*}

  \begin{align*}
    \frac{\SI{2}{\mega\bit\per\s}}{\SI{40}{\bit\per\s}} = 50000\ \text{Sender}
  \end{align*}

  \item
  \begin{align*}
    \begin{split}
      \text{ALOHA}\\
      \\
      \lim_{N \to \infty} \text{S} &= 1 \cdot e^{-2 \cdot 1} = 0,1353352832\\
      \lim_{N \to \infty} \text{S} &= 2 \cdot e^{-2 \cdot 4} = 0,03663127778\\
      \lim_{N \to \infty} \text{S} &= 10 \cdot e^{-2 \cdot 10} = 0,000000020611536
    \end{split}
    \begin{split}
      \text{S-ALOHA}\\
      \\
      \lim_{N \to \infty} \text{S} &= 1 \cdot e^{-1} = 0,3678794412\\
      \lim_{N \to \infty} \text{S} &= 2 \cdot e^{-2} = 0,2706705665\\
      \lim_{N \to \infty} \text{S} &= 10 \cdot e^{-10} = 0,0004539992976
    \end{split}
  \end{align*}

  Bei kleinen Gs ist der Unterschied der maximalen normalisierten Durchsätze noch nicht so dramatisch,
  da auch die Wahrscheinlichkeit für erfolgreiches Senden bei kleinen Gs näher beieinander liegt.

  \item
  \begin{align*}
    \text{Übertragungszeit} = \frac{\SI{1}{\km}}{\SI{250000}{\km\per\s}} = \SI{0,004}{\s} = \SI{4}{\ms}
  \end{align*}

  Die minimale Länge der Nachricht muss $2\tau$ Übertragungszeit haben:

  \begin{align*}
    \text{minimale Nachrichtengröße} = \SI{2}{\giga\bit\per\s} \cdot \SI{0,004}{\s} = \SI{8000000}{\bit} = \SI{8}{\mega\byte}
  \end{align*}
\end{enumerate}


\section*{2}

\begin{enumerate}[label=\alph*)]
  \item
  Methoden:
  \begin{itemize}
    \item BPSK (binary phase shift keyring)
    \item ASK (Amplitude-shift keying)
    \item FSK(Frequency Shift keyring)
  \end{itemize}

  Vorteile:
  \begin{itemize}
    \item Für die selbe „Bit Error Rate“ wird gegenüber BPSK die halbe Bandbreite gebraucht.
    \item Aufgrund der reduzierten Bandbreite ist auch die Informations-Übertragungsrate höher.
    \item Geringere Fehler-Wahrscheinlichkeit gegenüber ASK und FSK.
  \end{itemize}

  \addtocounter{enumi}{2}
  \item
  8-PSK ist eine Erweiterung von QPSK.

  QPSK-Methoden, die in der Praxis verwendet werden:
  \begin{itemize}
    \item DVB-S QPSK
    \item DVB-S2 QPSK
  \end{itemize}
\end{enumerate}


\section*{3}

\begin{enumerate}[label=\alph*)]
  \item
  Der Leitungscode legt bei der digitalen Telekommunikation fest, wie die zur
  Informationsübertragung genutzten Symbole auf der physischen Ebene übertragen werden.

  \addtocounter{enumi}{2}
  \item
  Vorteile vom Manchester-Encoding:
  \begin{itemize}
    \item Signal synchronisiert sich selbst.
    \item Geringe Fehlerrate und zuverlässige Übertragung durch „Self-Clocking“.
  \end{itemize}

  Nachteile vom Manchester-Encoding:
  \begin{itemize}
    \item Modulations-Bandbreite zweimal höher als bei NRZ.
    \item Relativ komplexes dekodieren.
  \end{itemize}

  Vorteile vom 4B5B-Encoding:
  \begin{itemize}
    \item Benötigt weniger Bandbreite gegenüber dem Manchester-Encoding.
  \end{itemize}
\end{enumerate}


\section*{4}

\begin{enumerate}[label=\alph*)]
  \item
  Jeder Kanal hat eine eigene Chip-Sequenz. Für eine 1 wird ein positiver Chip gesendet, und für eine 0 ein negativer. Der Empfänger muss den Chip zu jeder Empfangsfolge addieren, und kann somit das Signal demultiplexen.
\end{enumerate}

\end{document}
