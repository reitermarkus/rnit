\documentclass[t]{beamer}

\usepackage{lmodern}
\usepackage{textcomp}
\usetheme{metropolis} % https://github.com/matze/mtheme

\usepackage{multirow}

\usepackage[T1]{fontenc}
\usepackage[utf8]{inputenc}

\usepackage[ngerman]{babel}
\usepackage[ngerman]{isodate}

\usepackage{graphicx}
\graphicspath{{assets/}}

\usepackage[export]{adjustbox}

\title{SOCKSv5 (RFC 1928)}
\author{Michael Kaltschmid and Markus Reiter}
\date{\printdate{2017-06-13}}

\usepackage{ragged2e}

\begin{document}
  \RaggedRight

  \maketitle

  \begin{frame}{Überblick}
    \begin{itemize}
      \item Was sind Sockets?
      \item Aufbau von Sockets
      \item Typen: TCP und UDP
      \item Änderungen von v4 zu v5
      \item Ausblick
      \item Beispiel
    \end{itemize}
  \end{frame}

  \begin{frame}{Was sind Sockets?}
    \begin{itemize}
      \item vom Betriebssystem bereitgestellt
      \item Kommunikationsendpunkt
      \item Austausch von Daten
    \end{itemize}
  \end{frame}

  \begin{frame}{Ablauf: SOCKSv5 mit TCP}
    \begin{itemize}
      \item
        Client sendet Anfrage mit Versions-Identifikator \\
        und Authentifizierungmethoden:
        \begin{adjustbox}{max width=\textwidth}

          \begin{tabular}{|c|p{4cm}|p{4cm}|}
            \hline
            \texttt{VER} & \texttt{NMETHODS}                      & \texttt{METHODS} \\
            \hline
            \texttt{05}  & Anzahl der Oktetts in \texttt{METHODS} & 1 – 255 Methoden-Oktetts \\
            \hline
          \end{tabular}
        \end{adjustbox}
      \item
        Server wählt eine Methode in \texttt{METHODS} aus, und antwortet:
        \begin{adjustbox}{max width=\textwidth}
          \begin{tabular}{|c|p{8cm}|}
            \hline
            \texttt{VER} & \texttt{METHOD} \\
            \hline
            \texttt{05}  & Ein Oktett aus \texttt{METHODS}, oder \texttt{FF}, falls keine Methode akzeptabel ist. \\
            \hline
          \end{tabular}
        \end{adjustbox}
      \item
        Client schließt die Verbindung, falls er \texttt{FF} als Antwort erhält.
        Sonst wird eine Methoden-spezifische Authentifizierung gestartet.
    \end{itemize}
  \end{frame}

  \begin{frame}{Ablauf: SOCKSv5 mit TCP}
    \begin{itemize}
      \item
        Client sendet Request-Details:
        \begin{adjustbox}{max width=\textwidth}
          \begin{tabular}{|c|c|c|c|c|c|}
            \hline
            \texttt{VER} & \texttt{CMD}                & \texttt{RSV}         & \texttt{ATYP}            & \texttt{DST.ADDR} & \texttt{DST.PORT} \\
            \hline
            \texttt{05}  & \begin{tabular}{@{}l@{}}
                             \texttt{01} CONNECT \\ 
                             \texttt{02} BIND  \\ 
                             \texttt{03} UDP ASSOCIATE
                           \end{tabular}               & \texttt{00} RESERVED & \begin{tabular}{@{}l@{}}
                                                                                  \texttt{01} IPv4 \\ 
                                                                                  \texttt{03} Domain \\ 
                                                                                  \texttt{04} IPv6
                                                                                \end{tabular}            & Adresse           & Port \\
            \hline
          \end{tabular}
        \end{adjustbox}
      \item
        Server verarbeitet Request und antwortet:
        \begin{adjustbox}{max width=\textwidth}
          \begin{tabular}{|c|c|c|c|c|c|}
            \hline
            \texttt{VER} & \texttt{REP}                & \texttt{RSV}         & \texttt{ATYP}            & \texttt{DST.ADDR} & \texttt{DST.PORT} \\
            \hline
            \texttt{05}  & \begin{tabular}{@{}l@{}}
                             \texttt{00} succeeded \\
                             \texttt{01} general SOCKS server failure \\
                             \texttt{02} connection not allowed by ruleset \\
                             \texttt{03} Network unreachable \\
                             \texttt{04} Host unreachable \\
                             \texttt{05} Connection refused \\
                             \texttt{06} TTL expired \\
                             \texttt{07} Command not supported \\
                             \texttt{08} Address type not supported \\
                             \texttt{09} – \texttt{FF} unassigned \\
                           \end{tabular}               & \texttt{00} RESERVED & \begin{tabular}{@{}l@{}}
                                                                                  \texttt{01} IPv4 \\ 
                                                                                  \texttt{03} Domain \\ 
                                                                                  \texttt{04} IPv6
                                                                                \end{tabular}            & Adresse           & Port \\
            \hline
          \end{tabular}
        \end{adjustbox}
    \end{itemize}
  \end{frame}

\end{document}
