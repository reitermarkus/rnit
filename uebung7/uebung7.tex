% !TEX TS-program = xelatex
%
\documentclass[ngerman]{scrartcl}

\usepackage[ngerman]{babel}
\usepackage{amsmath}
\usepackage{enumitem}

\usepackage{siunitx}
\sisetup{per-mode = symbol, binary-units = true}
\DeclareSIUnit\mbyte{Byte}

\usepackage{graphicx}

\begin{document}

\section*{2}
	\begin{enumerate}[label=\alph*)]
    \item
      Base-64 wird hauptsächlich dazu verwendet, Daten zu übertragen, denn es könnte  sonst vorkommen, dass manche Zeichen als Kontrollzeichen behandelt werden.
    \item
    \ 
    \resizebox{\textwidth}{!}{
      \texttt{
        \begin{tabular}{|l|*{48}{c|}}
          \hline
          \textnormal{\textbf{ASCII}} & \multicolumn{8}{c|}{U} & \multicolumn{8}{c|}{I} & \multicolumn{8}{c|}{B} & \multicolumn{8}{c|}{K} & \multicolumn{8}{c|}{} & \multicolumn{8}{c|}{} \\
          \hline
          \textnormal{\textbf{Dezimal (Hex)}} & \multicolumn{8}{c|}{85 (0x55)} & \multicolumn{8}{c|}{73 (0x49)} & \multicolumn{8}{c|}{66 (0x42)} & \multicolumn{8}{c|}{75 (0x4B)} & \multicolumn{8}{c|}{} & \multicolumn{8}{c|}{} \\
          \hline
          \textnormal{\textbf{Binär}} & 0 & 1 & 0 & 1 & 0 & 1 & 0 & 1 & 0 & 1 & 0 & 0 & 1 & 0 & 0 & 1 & 0 & 1 & 0 & 0 & 0 & 0 & 1 & 0 & 0 & 1 & 0 & 0 & 1 & 0 & 1 & 1 & 0 & 0 & 0 & 0 & 0 & 0  & 0 & 0 & 0 & 0 & 0 & 0 & 0 & 0 & 0 & 0 \\
          \hline
          \textnormal{\textbf{Index}} & \multicolumn{6}{c|}{21} & \multicolumn{6}{c|}{20} & \multicolumn{6}{c|}{37} & \multicolumn{6}{c|}{2} & \multicolumn{6}{c|}{18}  & \multicolumn{6}{c|}{48} & \multicolumn{6}{c|}{0}& \multicolumn{6}{c|}{0} \\
          \hline
          \textnormal{\textbf{Base-64}} & \multicolumn{6}{c|}{V} & \multicolumn{6}{c|}{U} & \multicolumn{6}{c|}{l} & \multicolumn{6}{c|}{C} & \multicolumn{6}{c|}{S} & \multicolumn{6}{c|}{w} & \multicolumn{6}{c|}{=} & \multicolumn{6}{c|}{=} \\
          \hline
        \end{tabular}
      }
    }
    \addtocounter{enumi}{1}
    \item
    \begin{equation*}
      \frac{\SI{3072}{\byte} + (\SI{3072}{\byte}\ \%\ 3)}{3} \cdot 4 = \SI{4096}{\byte} \\
    \end{equation*}
    \begin{equation*}
      \SI{4096}{\byte} + 2 \cdot \frac{\SI{4096}{\byte}}{8} = \SI{5120}{\byte}
    \end{equation*}
  \end{enumerate}
\end{document}
