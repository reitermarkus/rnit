% !TEX TS-program = xelatex
%
\documentclass[ngerman]{scrartcl}

\usepackage[ngerman]{babel}
\usepackage{amsmath}
\usepackage{enumitem}

\usepackage{siunitx}
\sisetup{per-mode = symbol, binary-units = true}
\DeclareSIUnit\mbyte{Byte}

\begin{document}

\section*{1}
  \begin{enumerate}[label=\alph*)]
    \item
      Der Frequenzbereich zwischen \SI{883,6}{\mega\hertz} und \SI{889,8}{\mega\hertz} wird/wurde von der Mobilkom Austria verwendet.

      \begin{align*}
        f &= \SI{888}{\mega\hertz} \\
        \text{Wellenlänge}\ \lambda &= \frac{c}{f} \\
        \text{Wellenlänge}\ \lambda &= \frac{\SI{299792}{\kilo\meter\per\second}}{\SI{888}{\mega\hertz}} = \SI{33,76036036}{\centi\meter}
      \end{align*}

    \item
      \begin{align*}
        d + (k + \frac{1}{2}) \lambda = 0 \\
        d + (\lambda k + \frac{\lambda}{2}) &= 0 \\
        d + \frac{\lambda}{2} &= -\lambda k \\
        -\frac{d + \frac{\lambda}{2}}{\lambda} &= k \\
        -\frac{d}{\lambda} + \frac{1}{2} &= k \\
        -\frac{\SI{2}{\kilo\meter}}{\SI{33,76036036}{\centi\meter}} + \frac{1}{2} &= k \\
        5924,1073811176 + \frac{1}{2} &= k = 5924,6073811176
      \end{align*}

      Der Spiegel muss an der Position (1000; 5924,6073811176) aufgestellt werden.
  \end{enumerate}

\section*{2}
  \begin{enumerate}[label=\alph*)]
    \item
      Das Lesegerät sucht zuerst nach Tags anhand des höchstwertigen Bits, falls nur ein Tag antwortet, wurde er eindeutig indentifiziert, falls mehrere antworten, wird nach dem nächst-niedrigeren Bit gesucht. Dies wird solange wiederholt, bis nur noch ein einziger Tag antwortet oder das letzte Bit erreicht ist.
    \item
      Addressraum = $l = \SI{6}{\bit}$ \\
      Mögliche Tags = n = $2^l = 2^6 = 64$

      Laut der Komplexität von binären Suchbäumen ist die Anzahl der erwarteten Anfragen

      \begin{itemize}
        \item im Durchschnitt: $log(n) = log(64) \approx 1,8$ \\
        \item im schlimmsten Fall: $n = 64$
      \end{itemize}
  \end{enumerate}
\end{document}
